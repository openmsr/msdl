\section{Conclusions and future work}
In this project we started building the infrastructure for a molten salt data library. We have created a base standard for the data format and provided some tools to easily convert experimental results into the specified format as well as an easy to use tool for simple data analysis. This is however only the beginning and a lot of work is still needed before this project can be used in large scale projects in the likes of reactor design. For future work on the project we suggest to address the following issues:

\begin{enumerate}
    \item \textbf{Expand the data library} \\
    Over 100 experimental results have been converted to the proper data format and are ready to be analyzed by the tools, this only scratches the surface on how much data exists in literature. To represent a full fledged data library this needs to be expanded to encompass all important salt mixtures in the literature which could mean over a thousand sets. However the parser constructed for the project can easily add the data to the library thus expanding the data library is all about continuing the search through the scientific literature this project started.Obtaining the data from ORNL should be focused upon, as experimental data is very valuable. Another feature could be inclusion of Class-II and Class-III papers, when we have a robust Class-I data set available. The data library syntax will have to be updated on that. 
    \item \textbf{Eliminate dependencies} \\
    The Parser only accepts Microsoft Excel files for the input and the API only prints the output information in a Microsoft Excel format. This was done for convenience when starting the data collection and processing. As this project grows in scope however it's important that other alternatives are presented. This is to follow the design philosophy that the data library is accessible by everyone on multiple platforms.
    \item \textbf{Improve the mathematical modelling}
    Currently the program only supports one mathematical model when it comes to predicting physical properties of new salt mixtures which is the Regression Kriging model. This model is currently used while the dataset is still small and sparse but it will run into computational issues for larger data sets and is a blind blackbox model. The library of mathematical models to choose from should be expanded to allow for greater degree of data analysis. This could range from having alternative surrogate models such as Support Vector Regression or building a mechanistic model more grounded in the science of molten salts.
    \item \textbf{Improve user friendliness}\\
    Using the data library or any of the tools requires some basic knowledge of Python programming. However many of the fields experts may not be familiar with it or have time to learn the language. To make the library more accessible it may be necessary to develop a Graphical User Interface for the data library that retains all the features already implemented.
    
\end{enumerate}
Of course while tackling the aforementioned issues, all bugs reported should be fixed as soon as possible and code should be readily maintained to ensure that the project remains in a usable state for the foreseeable future.