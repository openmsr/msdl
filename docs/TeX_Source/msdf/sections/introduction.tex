\section{Introduction}

Molten salts generally describe a liquid obtained via fusion of one or more inorganic salts. Certain thermo-physical properties present in these salts make them attractive in scientific fields. Properties such as stability at high temperature ( $>$200 .C for high temperature applications) , low vapor pressure, high heat capacity, easy solubility of multiple compounds ; are looked into when molten salts are considered.\newline 

In the second half of 20th century, molten salts were being considered for Nuclear Applications. In the same period socio-political landscape resulted in a fairly large amount of work in short time. Aircraft Nuclear Propulsion (ANP) and Nuclear Energy for the Propulsion of Aircraft (NEPA) program by US Air Force were oriented towards using Nuclear technology for sustained air surveillance and as a viable nuclear strategic deterrent to Soviet capabilities. Oak Ridge National Laboratory (ORNL) was involved in research and also made the first Molten Salt Reactor (MSR) as part of the Aircraft Reactor Experiment. A fluoride salt of composition, NaF-ZrF4-UF4 (53-41-6) was used with beryllium oxide as a moderator. In 1961, the program was cancelled owing to extremely high cost of the project, without any viable results. The report stated that\textbf{\textit{15 years and about \$1 billion have been devoted to the attempted development of a nuclear-powered aircraft; but the possibility of achieving a militarily useful aircraft in the foreseeable future is still very remote.}} \cite{York} \newline  

However, this project confirmed the possible use of MSR for nuclear, and possibly civilian use. During the Molten-Salt Reactor Experiment (MSRE) which ran from 1960-1969, a 7.4 MW operational reactor was built and rigorous tests were carried out. During its successful runtime of almost 18000 hours of critical runtime (operating at 80 \% capacity) there were no major breakdowns, flaws or delays in form of downtime. \newline 
Since the late 70’s and towards the end of the 19th century. Nuclear technology was devoted to the conventional Uranium 235 based reactors. The events of Chernobyl and cold war arms race led to concerns amongst public and an outcry for discontinuation of commercial nuclear power plans. Latest studies have shown that out of currently used major sources of energy generation, nuclear is the safest. Under nominal circumstances, using nuclear energy for power generation would result in 0.07 death as compared with 24.62 via normal coal and 18.43 via oil \cite{markandya2007a}. The values are for production per TWh (Tera Watt hour). This clearly shows that the general perception on dangers of nuclear energy are short sighted and debatable. This is similar to the perception of climate change and global warming. One reason for this was discussed by George Marshall in his book \cite{marshall-g}. Using the same argument, it can be said that dangers of nuclear technology are very real and detailed in people’s mind, while that related to Coal or other conventional are relatively abstract and distant. Hence on a relativistic scale, the fallacious argument that nuclear power is on the extreme end of danger is ingrained within a high percentage of general population. \newline 


This short term pervasive mentality and public perception of the nuclear energy as a hazard led to the nuclear power phase out through Europe. After the end of cold war, focus was kept more on renewable energy research and virtually no research as conducted on MSR technology. This brings us to the reasoning and motivation behind initiation of this project.  \newline 

\subsection{Problem Statement and Proposed Solution}

Technologies for energy generation and energy storage are being evaluated in recent years. Molten salts are receiving special attention due to its aforementioned properties making it a suitable contender for heat transfer, coolant and heat storage substrate\cite{nunes2016a}\cite{serrano-2013a}. The important properties which are taken into consideration with renewed interest of molten salts are as follows.

\begin{itemize}
  \item high volumetric heat capacity,
 \item high boiling point and low vapor pressure,
 \item no undesirable chemical exothermic reactions between different zones of energy plants and power cycle coolants (core, heatexchange loop),
 \item optical transparency during inspection operations,
 \item ability to dissolve actinides,
 \item great insensitivity to radiations
 
\end{itemize}

In recent years, there has been a renewed interest in developing MSR and its related technology. This push is partly because of the limitations seen in renewable energy and the need for a suitable sustainable alternative; and partly because of the perception change on nuclear energy and availability of funding to model and develop the technology. Technology like Concentrated Solar Plant (CSP) are have also gained traction, where integral storage is a component of the process. Molten salts are rigorously studied and their property such as volumetric absorption are deciding factors in the development of the same\cite{slocum2011a}. \newline

\subsubsection{\textbf{Problem Statement}}
Looking into these technologies, one thing gradually becomes clear. Molten salts will be used in a wide variety of applications within nuclear, solar and chemical industry. Thermo-physical and chemical property data for a wide variety of eutectic salt mixtures are necessary to study and evaluate further. However, a literature survey of experimental properties on molten salts found that the data was spread over 30 years in multiple papers. This gave a problem statement in the form, \textbf{\textit{while recently garnered interest in technologies have focus on study of molten salt properties, the dearth and difficulty in finding data in suitable format can create a possible hindrance for researchers intending to work on this subject}}. This project is aimed at starting to solve this problem.  \newline

\subsubsection{Proposed Solution}
Solution Strategy was discussed with project supervisors who had relevant knowledge in the field and who were industrially established to give suitable advice. The proposed solution was \textbf{\textit{creation of an online data library which can act as a single major source for obtaining thermo-physical and chemical properties.}} The data would on an open source platform in a suitable format for wide accessibility. The methodology which is used in creation of this library is in two fold step. 
\begin{itemize} 
\item Step-1 Literature sweep of papers in which experimental property of molten salts are given. The data from papers should be converted into an efficient sustainable data format with human readable meta-data where the source is listed.
\item Step-2 Develop an Application Program Interface (API) that can filter and extract relevant data and perform rudementary data processing for modelling purposes.
\end{itemize}

Final product was planned to have a robust programming interface to carry out analysis of various properties based on the library created as well as the possibility of predicting Molten Salt properties for modelling work. The main aim of this project is to help academic people who are starting to work on Molten Salts via single source to obtain and analyze their thermo-physical properties. Other advantages of this project are that due to It being open source, contribution to this project becomes easy and in a circular fashion, available to everyone. \newline


\subsubsection{Report Structure}
The report structure is also kept in the two parts. First part , data gathering, as the names suggests describes the process through which experimental data relevant to molten salts was obtained and converted to excel files. An intermediate step of data extraction to excel files was used because a large proportion of papers with experimental data were quite old, making it difficult to convert them to API. The second part describes the technical aspects of the programming. It describes the features developed and the logic behind it. This section contains user guide as an explanation for new users on getting started.  \newpage
